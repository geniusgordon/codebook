\documentclass{article}
 
\usepackage[landscape,margin=1cm,a4paper]{geometry}
\usepackage{multicol}
\usepackage{listings}

\lstset{
	language=C++,
	columns=flexible,
	tabsize=4
}
 
\begin{document}
\begin{multicols}{2}
\ttfamily
\begin{lstlisting}
#include <iostream>
#include <vector>
using namespace std;

struct SegmentTreeNode {
	SegmentTreeNode *left, *right;
	int L, R;
	int value;

	SegmentTreeNode() {
		left = right = NULL;
		L = R = -1;
		value = 0;
	}
};

struct SegmentTree {
	vector<int> array;
	SegmentTreeNode* root;

	SegmentTree() {
		root = new SegmentTreeNode();
	}

	void destroy() {
		_destroy(root);
	}

	void _destroy(SegmentTreeNode* node) {
		if (!node)
			return;
		_destroy(node->left);
		_destroy(node->right);
		delete node;
	}

	void build() {
		_build(0, array.size(), root);
	}

	int query(int x, int y) {
		return _query(x, y, root);
	}

\end{lstlisting}
\columnbreak
\begin{lstlisting}
	void _build(int L, int R, SegmentTreeNode* node) {
		node->L = L;
		node->R = R;

		if (L+1 == R) {
			node->value = array[L];
			return;
		}

		int M = (L + R) / 2;

		if (!node->left) node->left = new SegmentTreeNode();
		if (!node->right) node->right = new SegmentTreeNode();

		_build(L, M, node->left);
		_build(M, R, node->right);

		node->value = node->left->value + node->right->value;
	}

	int _query(int x, int y, SegmentTreeNode* node) {
		if (!node)
			return 0;
		if (y <= node->L || node->R <= x)
			return 0;
		if (x <= node->L && node->R <= y)
			return node->value;

		int a = _query(x, y, node->left);
		int b = _query(x, y, node->right);

		return a + b;
	}
};
\end{lstlisting}
\newpage
\begin{lstlisting}
#include <iostream>
#include <algorithm>
#include <vector>
#define N 10000000
using namespace std;

vector<long long> array;
long long dp[N], len[N], pos[N];
const long long INF = 0x3fffffffffffffff;

vector<long long> findLIS()
{
	int n = array.size();

	for (int i = 0; i <= n; i++)
		len[i] = INF;
	for (int i = 0; i < n; i++) {
		int k = lower_bound(len+1, len+n+1, array[i]) - len;
		dp[i] = k;
		len[k] = array[i];
	}
	long long L = 0;
	for (int i = 0; i < n; i++)
		L = max(L, dp[i]);

	vector<long long> lis = vector<long long>(L, 0);
	int tmp = L;
	for (int i = n-1; i >= 0; i--) {
		if (dp[i] == tmp) {
			lis[tmp-1] = array[i];
			tmp--;
		}
	}
	return lis;
}
\end{lstlisting}
\columnbreak
\begin{lstlisting}
int LCIS(vector<int>& A, vector<int>& B)
{
	vector<int> C(B.size(), 0);

	for (int i = 0; i < A.size(); i++) {
		int cur = 0;
		for (int j = 0; j < B.size(); j++) {
			if (A[i] == B[j] && cur+1 > C[j]) {
				C[j] = cur + 1;
			} else if (A[i] > B[j] && cur < C[j]) {
				cur = C[j];
			}
		}
	}
	int ret = 0;
	for (int i = 0; i < C.size(); i++) {
		ret = max(ret, C[i]);
	}
	return ret;
}
\end{lstlisting}
\newpage
\begin{lstlisting}
#include <iostream>
#include <vector>
#include <cstdio>
#include <cstring>
#define N 15
using namespace std;

const int INF = 1000;
int graph[N][N];
int dp[1<<N][N];

int _TSP(int mask, int x, int n)
{
	if (!(mask & (mask-1)))
		return 0;
	if (dp[mask][x] != -1)
		return dp[mask][x];
	int res = INF;
	for (int i = 0; i < n; i++) {
		if (i == x)
			continue;
		if (mask & (1<<i)) {
			int tmp = _TSP(mask^(1<<x), i, n) + graph[x][i];
			res = min(res, tmp);
		}
	}
	return dp[mask][x] = res;
}

int TSP(int n)
{
	int res = INF;
	int e = (1<<n) - 1;
	for (int i = 0; i < n; i++) {
		memset(dp, -1, sizeof(dp));
		int tmp = _TSP(e, i, n);
		res = min(res, tmp);
	}
	return res;
}
\end{lstlisting}
\columnbreak
\begin{lstlisting}
#include <vector>
#include <cstring>
#define N 1000
using namespace std;

vector<int> graph[N];
int match[N];
bool vst[N];

bool _bipartite(int x)
{
	for (int i = 0; i < graph[x].size(); i++) {
		int y = graph[x][i];
		if (!vst[y]) {
			vst[y] = true;
			if (match[y] == -1 || _bipartite(match[y])) {
				match[y] = x;
				return true;
			}
		}
	}
	return false;
}

void bipartite(int n)
{
	memset(match, -1, sizeof(match));
	for (int i = 0; i < n; i++) {
		memset(vst, false, sizeof(vst));
		_bipartite(i);
	}
}
\end{lstlisting}
\newpage
\begin{lstlisting}
vector<int> adj[N];
int clk[N], low[N], int t;

vector<int> ap;
vector< pair<int, int> > bridge;

void dfs(int cur, int parent)
{
	int child = 0;
	bool flag = false;

	low[cur] = clk[cur] = t;
	t++;
	
	for (int i = 0; i < adj[cur].size(); i++) {
		int next = adj[cur][i];
		if (!clk[next]) {
			child++;
			dfs(next, cur);
			low[cur] = min(low[cur], low[next]);

			if (low[next] >= clk[cur])
				flag = true;
			if (low[next] > clk[cur])
				bridge.push_back(make_pair(cur, next));
		} else if (next != parent) {
			low[cur] = min(low[cur], clk[next]);
		}
	}
	if (parent == -1 && child >= 2)
		ap.push_back(cur);
	else if (parent != -1 && flag)
		ap.push_back(cur);
}

void tarjan(int n)
{
	t = 1;
	ap.clear();
	bridge.clear();
	memset(clk, 0, sizeof(clk));
	memset(low, 0, sizeof(low));
	dfs(0, -1);
}
\end{lstlisting}
\columnbreak
\begin{lstlisting}
#include <iostream>
#include <vector>
using namespace std;

vector<int> build_fail_function(string S)
{
	int len = S.length(), cur;
	vector<int> fail = vector<int>(len, 0);

	cur = fail[0] = -1;
	for (int i = 1; i < len; i++) {
		while (cur != -1 && S[cur+1] != S[i]) {
			cur = fail[cur];
		}
		if (S[cur+1] == S[i])
			cur++;
		fail[i] = cur;
	}
	return fail;
}

vector<int> match(string A, string B)
{
	int lenA = A.length(), lenB = B.length();
	int cur = -1;
	vector<int> pos;
	vector<int> fail = build_fail_function(B);
	for (int i = 0; i < lenA; i++) {
		while (cur != -1 && B[cur+1] != A[i]) {
			cur = fail[cur];
		}
		if (B[cur+1] == A[i])
			cur++;
		if (cur == lenB-1) {
			pos.push_back(i);
			cur = fail[cur];
		}
	}
	return pos;
}
\end{lstlisting}
\newpage
\begin{lstlisting}
#include <iostream>
#include <cstdio>
#include <cstring>
using namespace std;

struct TrieNode {
	TrieNode* to[26];
	bool word;

	TrieNode() {
		word = false;
		memset(to, 0, sizeof(to));
	}
};

struct Trie {
	TrieNode* root;

	Trie () {
		root = new TrieNode();
	}

	~Trie() {
		freeNode(root);
	}

	void freeNode(TrieNode* ptr) {
		for (int i = 0; i < 26; i++) {
			if (ptr->to[i])
				freeNode(ptr->to[i]);
		}
		delete ptr;
	}

\end{lstlisting}
\columnbreak
\begin{lstlisting}
	void insert(string& s) {
		TrieNode* ptr = root;
		for (int i = 0; i < s.length(); i++) {
			if (!ptr->to[s[i]-'a'])
				ptr->to[s[i]-'a'] = new TrieNode();
			ptr = ptr->to[s[i]-'a'];
		}
		ptr->word = true;
	}

	bool contain(string& s) {
		TrieNode* ptr = root;
		for (int i = 0; i < s.length(); i++) {
			if (!ptr->to[s[i]-'a'])
				return false;
			ptr = ptr->to[s[i]-'a'];
		}
		return ptr->word;
	}
};
\end{lstlisting}
\newpage
\begin{lstlisting}
#include <iostream>
#include <cstdio>
#include <cstring>
#include <vector>
#include <queue>
#include <stack>
#define N 2000
using namespace std;

vector <int> adj[N]; //adj list
vector <int> rev[N]; //adj list reverse
vector <int> finish;  //the order of leaving 1st dfs
vector <int> scc[N];
int group[N];
bool vst[N];

void dfs1(int x) //first dfs to get the order of leaving dfs
{
	vst[x] = true;
	for (int i = 0; i < adj[x].size(); i++)
		if (!vst[adj[x][i]])
			dfs1(adj[x][i]);
			
	finish.push_back(x);
}

void dfs2(int x, int c) //use the reverse adj list to dfs
{
	scc[c].push_back(x);
	group[x] = c;
	
	vst[x] = true;
	for (int i = 0; i < rev[x].size(); i++)
		if (!vst[rev[x][i]])
			dfs2(rev[x][i], c);
}

void find_ssc(int n)
{
		memset(vst, false, sizeof(vst));
		for (int i = 1; i <= n; i++)// first dfs
			if (!vst[i])
				dfs1(i);

		int c = 0;
		memset(vst, false, sizeof(vst));
		for (int i = finish.size()-1; i >= 0; i--) //second dfs using the order of finish from the end
			if (!vst[finish[i]])
				dfs2(finish[i], c++);
}
\end{lstlisting}
\end{multicols}
 
\end{document}

